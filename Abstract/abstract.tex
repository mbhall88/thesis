% ************************** Thesis Abstract *****************************
% Use `abstract' as an option in the document class to print only the titlepage and the abstract.
\begin{abstract}
% briefly describe the wider context
Improving our ability to investigate variation within bacterial genomes has many important implications. The advent of Nanopore sequencing technology has shown great promise for being able to interrogate bacterial genomes with greater speed and mobility along with a lower resource requirement to get started. 
% present a summary of current status
Traditionally, methods developed for human genomes have been applied to bacteria even though their genomes are structured very differently. Genome graphs are a different approach to genomic analysis that have seen significant development recently. Rather than a single reference genome that represents a species, a collection of known variation is used. 
Bacteria have a pan-genome, whereby some genes are shared by all members of the species (core) and others are only found in a small number (accessory). 
% state main aim of each chapter
Previous work in the lab developed a data structure and associated algorithms for representing bacterial genomes as a collection of graphs, where each graph represents a locus such as a gene. One limitation of this approach is an inability to identify when novel variation is present. That is, it can only describe genomes as a mosaic of the known variation. We extend this method by adding the ability to discover novel variation by performing local assembly with de Bruijn graphs in regions we suspect contain variation not within the graph. We then validate this approach for both Illumina and Nanopore data and develop a comprehensive framework for evaluating variant calls from a genome graph against traditional single reference variant callers.
\mtb{} is one pathogen of particular Public Health interest as it kills more people each year than any other pathogen. Whole-genome Illumina sequencing is routinely being used to provide Public Health authorities in high-income, low-burden countries with information about likely transmission clusters and prediction of drug resistance. Given many high-burden settings have reduced resources compared to their high-income counterparts, Nanopore sequencing presents a potential boon for providing Public Health authorities with with important information like potential transmission clusters and drug resistance predictions. There has been some exploration of Nanopore's ability to provide this information for TB, but no detailed exploration. We firstly show Nanopore data is capable of providing transmission clusters inline with those from Illumina data and investigate the potential for genome graphs to help with this task. Next, we turn our attention to drug resistance prediction, and again show that Nanopore data provides predictions in concordance with Illumina. Additionally, we layout and evaluate a framework for using genome graphs to predict drug resistance with the advantage of being able to flag novel variation is genes implicated in resistance to antimicrobial drugs.
Lastly, we pave the way for further improvements of Nanopore applications in TB by training a species-specific basecalling model and show that this model provides a dramatic improvement in read- and assembly-level accuracy.
% state impact in wider context
Taken together this work opens the door for Nanopore-based TB public health and lays the foundations for using genome graphs to predict drug resistance in bacterial genomes.

\end{abstract}
