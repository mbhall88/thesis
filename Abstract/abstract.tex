% ************************** Thesis Abstract *****************************
% Use `abstract' as an option in the document class to print only the titlepage and the abstract.
\begin{abstract}

Variation in bacterial genomes has important implications for global public health. For example, the ability to identify transmission events and predict antimicrobial drug resistance (AMR) depends on such differences. A single reference genome is often used as a template to find variation. However, bacterial species have a pan-genome, whereby some genes are common to all and others only appear once. Genome graphs are a novel approach to genomic analysis that suit pan-genomes. They use a collection of known genetic variation instead of a single reference. Previous genome graph work has developed methods for representing bacterial genomes as a set of unordered graphs, where each graph represents a locus such as a gene. One limitation, though, is that it can only describe genomes as a mosaic of known variation. Finding novel variation is central to many applications, such as those mentioned earlier. Therefore, in the first chapter, we extend this genome graph method to allow \textit{de novo} variant discovery and develop a process for evaluating graph-based variant calls.

Many public health authorities use whole-genome sequencing (WGS) to surveil bacterial pathogens. \textit{Mycobacterium tuberculosis} - the causative agent of tuberculosis (TB) - is one species of particular public health interest as it kills more people each year than any other pathogen. Some countries use Illumina WGS to assess TB patients for transmission clusters and AMR, although this is not ideal in some high-burden settings, which often have reduced resources. Nanopore sequencing presents a potential boon for providing the same information as Illumina with greater speed and lower resource requirements. However, there has been no detailed study of Nanopore’s ability to provide this information for TB, which is the aim of the remaining three chapters. First, we show that Nanopore data can identify transmission clusters consistent with those from Illumina and investigate the potential for genome graphs to help with this task. Second, we demonstrate that Nanopore provides AMR predictions concordant with Illumina and introduce a framework for using genome graphs to improve current methods. Lastly, we train a Nanopore basecalling model for TB with considerable read- and assembly-level accuracy improvement.

In summary, this thesis improves genome graph variant-calling in bacterial genomes and opens the door for Nanopore-based TB public health applications.

\end{abstract}

