%!TEX root = ../thesis.tex
%*******************************************************************************
%****************************** Second Chapter *********************************
%*******************************************************************************

\chapter{My second chapter}

\ifpdf
    \graphicspath{{Chapter2/Figs/Raster/}{Chapter2/Figs/PDF/}{Chapter2/Figs/}}
\else
    \graphicspath{{Chapter2/Figs/Vector/}{Chapter2/Figs/}}
\fi

\subsection{Nanopore data pre-processing}

Basecalling and demultiplexing.

\subsection{Assembly}
\begin{markdown}

Samples with greater than 30x coverage across all three sequencing technologies were chosen to produce high-quality assemblies. In total, this left us with 9 Malagasy samples. We compare five assemblers and select the best for each sample. The reason for this comparison is that different assembly algorithms can produce quite varied results depending on sequencing technology used, species, or computational resource availability(CITE).  
The assembly tools used are Canu, Flye, Unicycler, HASLR, and Spades(CITE \& VERSION). HASLR and Unicycler are hybrid assemblers that take Illumina reads along with one long-read file, although Unicycler does not require both. Spades is also a hybrid assembler but takes an arbitrary number of different sequencing technologies. Canu and Flye are both long-read-only assemblers.  

The entire assembly pipeline was orchestrated using the workflow management system Snakemake(CITE). An overview of the entire pipeline is shown in (FIGURE). The first step is trimming of adapter sequences in the Illumina reads using Trimmomatic(CITE). Two assemblies were produced for each sample - one for each long-read technology. The exception to this was Spades, for which there is just one assembly for each sample, as it accepts all reads simultaneously.


\missingfigure{Snakemake assembly pipeline DAG}



\end{markdown}